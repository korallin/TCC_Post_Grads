% ----------------------------------------------------------
% Introdução (exemplo de capítulo sem numeração, mas presente no Sumário)
% ----------------------------------------------------------
\chapter{Introdução}
% ----------------------------------------------------------

Atualmente na Justiça Federal do Rio Grande do Norte (JFRN) existe a necessidade de se monitorar o que entra nas Varas espalhadas pelo Estado, isso é importante para se entender qual tipo de demanda é mais comum, assim, distribuindo melhor os recursos disponíveis. Uma das estratégias usadas para entender o que está acontecendo melhor nas Varas é ver as demandas repetitivas, esse tipo de demanda pode atrapalhar o fluxo de trabalho por que, frequentemente, acontece em grande volume demandando bastante esforço para ser resolvido. Então, é interessante que haja uma forma de visualizar algum tipo de demanda que distoa das outras ao longo de determinado período de tempo, mensalmente digamos, para que a Justiça consiga agir mais rapidamente, impedindo que essa massa de processos cresça demasiadamente. A ideia nesse trabalho é criar uma forma de visualizar esses dados de forma simples, fácil e leve, e que não seja necessário ser de T.I. para mexer nesse painel.

% ---
% Capitulo com exemplos de comandos inseridos de arquivo externo 
% ---
\chapter{Ferramentas de criação de painéis}\label{cap_trabalho_academico}

Antes apresentar os detalhes do trabalho, é necessário explicar alguns conceitos que serão usados ao longo dos capítulos, entre eles o que é \textit{Business Intelligence}(BI) e o que são painéis BI.

De acordo com ADFASDFASDFASDF Business Intelligence é FASDFASDFSAF ASDFSF ASD SADF

E OS PAINÉIS BI SÃO A business dashboard is an information management tool that tracks and simplifies complex data sets and leverages data visualizations, allowing users to gain quick insight into current performance. It's also commonly known as a business intelligence dashboard or BI dashboard.

\chapter{Dash e Plotly}\label{cap_trabalho_academico}

\chapter{Limpeza de dados}\label{cap_trabalho_academico}

% ---
% ----------------------------------------------------------
% Finaliza a parte no bookmark do PDF
% para que se inicie o bookmark na raiz
% e adiciona espaço de parte no Sumário
% ----------------------------------------------------------
\phantompart

% Conclusão
\chapter{Conclusão}
\lipsum[31-33]