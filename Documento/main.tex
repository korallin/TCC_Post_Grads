% ---------------------------------------------------------------------
% abnTeX2: Modelo de Trabalho Academico (tese de doutorado, dissertacao
% de mestrado e trabalhos monograficos em geral) em conformidade com 
% ABNT NBR 14724:2011: Informacao e documentacao - Trabalhos academicos 
% Apresentacao
% --------------------------------------------------------------------

\documentclass[
    12pt,				% tamanho da fonte
	oneside,            % impressao em um único lado
	a4paper,			% tamanho do papel
	english,			% idiomas adicionais
	french,
	spanish,
	brazil				% o último idioma é o principal do documento
	]{abntex2}
	
% Customizaçõs para Residência TI
\usepackage{residencia-ti}

% Pacotes básicos 
\usepackage{lmodern}			% Usa a fonte Latin Modern			
\usepackage[T1]{fontenc}		% Selecao de codigos de fonte.
\usepackage[utf8]{inputenc}		% Codificacao do documento (conversão automática dos acentos)

\usepackage{color}				% Controle das cores
\usepackage[pdftex]{graphicx}	% Inclusão de gráficos
\usepackage{indentfirst}		% Indenta o primeiro parágrafo de cada seção
\usepackage{microtype} 			% Melhorias de justificação
\usepackage{pdftexcmds}         % Condicional
		
% Pacotes adicionais, usados apenas no âmbito do Modelo Canônico do abnteX2
\usepackage{lipsum}				% Geração de dummy text

% Pacotes de citações
\usepackage[brazilian,hyperpageref]{backref}	% Paginas com as citações na bibl
\usepackage[alf]{abntex2cite}	                % Citações padrão ABNT

% Configurações do pacote backref
% Usado sem a opção hyperpageref de backref
\renewcommand{\backrefpagesname}{Citado na(s) página(s):~}
% Texto padrão antes do número das páginas
\renewcommand{\backref}{}
% Define os textos da citação
\renewcommand*{\backrefalt}[4]{
	\ifcase #1 %
		Nenhuma citação no texto.%
	\or
		Citado na página #2.%
	\else
		Citado #1 vezes nas páginas #2.%
	\fi}%

% Informações de dados da Instituição
\providecommand{\imprimiruniversidade}{}
\newcommand{\universidade}[1]{\renewcommand{\imprimiruniversidade}{#1}}
\providecommand{\imprimircentro}{}
\newcommand{\centro}[1]{\renewcommand{\imprimircentro}{#1}}
\providecommand{\imprimirdepartamento}{}
\newcommand{\departamento}[1]{\renewcommand{\imprimirdepartamento}{#1}}
\providecommand{\imprimirprograma}{}
\newcommand{\programa}[1]{\renewcommand{\imprimirprograma}{#1}}

\titulo{Painel em Python para o Centro de Inteligência}
\autor{Kallil de Araújo Bezerra}
\local{Natal-RN, Brasil}
\data{\the\year}
\orientador{Elias Jacob}
%\coorientador{Nome do coorientador}
\universidade{Universidade Federal do Rio Grande do Norte}
\centro{Instituto Metrópole Digital}
\programa{Programa de Residência em Tecnologia da Informação}
\tipotrabalho{Trabalho de Conclusão de Curso}

% O preambulo deve conter o tipo do trabalho, o objetivo, % o nome da instituição e a área de concentração 
\preambulo{\imprimirtipotrabalho~apresentado ao~\imprimirprograma~do~\imprimircentro~ da~\imprimiruniversidade~como requisito parcial para a obtenção do título de Especialista em Tecnologia da Informação. Área de Concentração: 
}

% Configurações de aparência do PDF final

% alterando o aspecto da cor azul
\definecolor{blue}{RGB}{41,5,195}

% informações do PDF
\makeatletter
\hypersetup{
    %pagebackref=true,
	pdftitle={\@title}, 
	pdfauthor={\@author},
    pdfsubject={\imprimirpreambulo},
	colorlinks=true,       		% false: boxed links; true: colored links
    linkcolor=black,          	% color of internal links
    citecolor=black,        	% color of links to bibliography
    filecolor=magenta,     		% color of file links
	urlcolor=blue,
	bookmarksdepth=4
}
\makeatother

% Posiciona figuras e tabelas no topo da página quando adicionadas sozinhas
% em um página em branco
\makeatletter
\setlength{\@fptop}{5pt} % Set distance from top of page to first float
\makeatother

% Possibilita criação de Quadros e Lista de quadros.
\newcommand{\quadroname}{Quadro}
\newcommand{\listofquadrosname}{Lista de quadros}

\newfloat[chapter]{quadro}{loq}{\quadroname}
\newlistof{listofquadros}{loq}{\listofquadrosname}
\newlistentry{quadro}{loq}{0}

% Configurações para atender às regras da ABNT
\setfloatadjustment{quadro}{\centering}
\counterwithout{quadro}{chapter}
\renewcommand{\cftquadroname}{\quadroname\space} 
\renewcommand*{\cftquadroaftersnum}{\hfill--\hfill}
\setfloatlocations{quadro}{hbtp}

% Espaçamentos entre linhas e parágrafos 
% O tamanho do parágrafo é dado por:
\setlength{\parindent}{1.3cm}

% Controle do espaçamento entre um parágrafo e outro:
\setlength{\parskip}{0cm}  % tente também \onelineskip

% Compila o indice
\makeindex

% Início do documento
\begin{document}

% Seleciona o idioma do documento (conforme pacotes do babel)
%\selectlanguage{english}
\selectlanguage{brazil}

% Retira espaço extra obsoleto entre as frases.
\frenchspacing 

% ELEMENTOS PRÉ-TEXTUAIS
% \pretextual

% Capa
\imprimircapa

% Folha de rosto
% (o * indica que haverá a ficha bibliográfica)
\imprimirfolhaderosto*

% Inserir a ficha bibliografica
% A biblioteca da sua universidade lhe fornecerá um PDF
% com a ficha catalográfica definitiva após a defesa do trabalho. Quando estiver
% com o documento, salve-o como PDF no diretório do seu projeto e substitua todo
% o conteúdo de implementação deste arquivo pelo comando abaixo:
%
% \begin{fichacatalografica}
%     \includepdf{fig_ficha_catalografica.pdf}
% \end{fichacatalografica}

% Inserir folha de aprovação
% Isto é um exemplo de Folha de aprovação, elemento obrigatório da NBR
% 14724/2011 (seção 4.2.1.3). Você pode utilizar este modelo até a aprovação
% do trabalho. Após isso, substitua todo o conteúdo deste arquivo por uma
% imagem da página assinada pela banca com o comando abaixo:
%
% \begin{folhadeaprovacao}
% \includepdf{folhadeaprovacao_final.pdf}
% \end{folhadeaprovacao}
%
\begin{folhadeaprovacao}

  \begin{center}
    {\ABNTEXchapterfont\large\imprimirautor}

    \vspace*{\fill}\vspace*{\fill}
    \begin{center}
      \ABNTEXchapterfont\bfseries\Large\imprimirtitulo
    \end{center}
    \vspace*{\fill}
    
    \hspace{.45\textwidth}
    \begin{minipage}{.5\textwidth}
        \imprimirpreambulo
    \end{minipage}%
    \vspace*{\fill}
   \end{center}
        
   Trabalho aprovado. \imprimirlocal, \today:

   \assinatura{\textbf{\imprimirorientador} \\ Orientador}
   %\assinatura{\textbf{\imprimircoorientador} \\ Coorientador} 
   \assinatura{\textbf{Professor} \\ Examinador}
   \assinatura{\textbf{Professor} \\ Examinador}
      
   \begin{center}
    \vspace*{0.5cm}
    {\imprimirlocal}
    \par
    {\imprimirdata}
    \vspace*{1cm}
  \end{center}
\end{folhadeaprovacao}

\setlength{\parskip}{0cm}  % tente também \onelineskip

% Dedicatória
\begin{dedicatoria}
   \vspace*{\fill}
   \centering
   \noindent
   \textit{Dedicatória.} \vspace*{\fill}
\end{dedicatoria}

% Agradecimentos
\begin{agradecimentos}
Agradeço à minha família e também aos meus colegas residentes da Justiça Federal do Rio Grande do Norte que sempre me ajudaram e com quem eu tive a grande honra de aprender tanto.
\end{agradecimentos}

% Epígrafe
\begin{epigrafe}
    \vspace*{\fill}
	\begin{flushright}
		Epígrafe
	\end{flushright}
\end{epigrafe}

% RESUMOS
% Resumo em português
\setlength{\absparsep}{18pt} % ajusta o espaçamento dos parágrafos do resumo
\begin{resumo}
\vspace{\onelineskip}
Há algumas décadas organizações governamentais e não governamentais já tomam decisões usando dados colhidos através de pesquisas internas e externas, porém, a capacidade de trabalhar com dados hoje é muito superior aos anos passados, atualmente existem análises mais poderosas que podem ser feitas graças aos computadores mais potentes e a infraestrutura de Tecnologia de Informação mais madura. Além disso, com o mundo digital também veio a necessidade de aumentar a velocidade da tomada de decisões, e nesse contexto a visualização de dados se torna importante porque pode mostrar de forma rápida o que está acontecendo com as variáveis de interesse, e como elas se comportam em relação àos outros pontos de interesse.

\noindent\textbf{Palavras-chave}: Visualização de dados. Análise de dados. Estatística.
\end{resumo}

% Resumo em inglês
\begin{resumo}[Abstract]
\vspace{\onelineskip}
\begin{otherlanguage*}{english}
O resumo em língua estrangeira (em inglês Abstract, em espanhol Resumen, em francês Résumé) é uma versão do resumo escrito na língua vernácula para idioma de divulgação internacional. Ele deve apresentar as mesmas características do anterior (incluindo as mesmas palavras, isto é, seu conteúdo não deve diferir do resumo anterior), bem como ser seguido das palavras representativas do conteúdo do trabalho, isto é, palavras-chave e/ou descritores, na língua estrangeira. Embora a presente especificação considere o inglês como língua estrangeira (o mais comum), não fica impedida a adoção de outras línguas (a exemplo de espanhol ou francês) para redação do resumo em língua estrangeira.

\noindent\textbf{Keywords}: Data visualization. Data analysis. Statistics.
\end{otherlanguage*}
\end{resumo}

% Resumo em francês 


% Inserir lista de ilustrações
\pdfbookmark[0]{\listfigurename}{lof}
\listoffigures*
\clearpage

% Inserir lista de quadros
\pdfbookmark[0]{\listofquadrosname}{loq}
\listofquadros*
\clearpage

% Inserir lista de tabelas
\pdfbookmark[0]{\listtablename}{lot}
\listoftables*
\clearpage

% Inserir lista de abreviaturas e siglas
\begin{siglas}
  \item[IMD] Instituto Metrópole Digital
  \item[UFRN] Universidade Federal do Rio Grande do Norte
  \item[JFRN] Justiça Federal do Rio Grande do Norte
\end{siglas}

% Inserir lista de símbolos
\begin{simbolos}
  \item[$\Gamma$] Letra grega Gama
  \item[$\Lambda$] Lambda
  \item[$\zeta$] Letra grega minúscula zeta
\end{simbolos}

% Inserir o sumario
\pdfbookmark[0]{\contentsname}{toc}
\tableofcontents*
\clearpage


% ELEMENTOS TEXTUAIS
\textual

% CONTEÚDO (conteudo.tex)
% Pode-se ter múltiplos arquivos, um para cada capítulo
% ----------------------------------------------------------
% Introdução (exemplo de capítulo sem numeração, mas presente no Sumário)
% ----------------------------------------------------------
\chapter{Introdução}
% ----------------------------------------------------------

Atualmente na Justiça Federal do Rio Grande do Norte (JFRN) existe a necessidade de se monitorar o que entra nas Varas espalhadas pelo Estado, isso é importante para se entender qual tipo de demanda é mais comum, assim, distribuindo melhor os recursos disponíveis. Uma das estratégias usadas para entender o que está acontecendo melhor nas Varas é ver as demandas repetitivas, esse tipo de demanda pode atrapalhar o fluxo de trabalho por que, frequentemente, acontece em grande volume demandando bastante esforço para ser resolvido. Então, é interessante que haja uma forma de visualizar algum tipo de demanda que destoa das outras ao longo de determinado período de tempo, mensalmente digamos, para que a Justiça consiga agir mais rapidamente, impedindo que essa massa de processos cresça demasiadamente. A ideia nesse trabalho é criar uma forma de visualizar esses dados de forma simples, fácil e leve, e que não seja necessário ser de T.I. para mexer nesse painel. \cite{ibge1993}

% ---
% Capitulo com exemplos de comandos inseridos de arquivo externo 
% ---
\chapter{Ferramentas de criação de painéis}\label{cap_trabalho_academico}

Antes apresentar os detalhes do trabalho, é necessário explicar alguns conceitos que serão usados ao longo dos capítulos, entre eles o que é \textit{Business Intelligence}(BI) e o que são painéis BI.

De acordo com ADFASDFASDFASDF Business Intelligence é FASDFASDFSAF ASDFSF ASD SADF

E OS PAINÉIS BI SÃO A business dashboard is an information management tool that tracks and simplifies complex data sets and leverages data visualizations, allowing users to gain quick insight into current performance. It's also commonly known as a business intelligence dashboard or BI dashboard.

\chapter{Ferramentas de dashboard}
\begin{itemize}
	\item qlikview
	
	\item pentaho
	
	\item powerbi
	
	\item metabase
\end{itemize}

Razões para escolher o Python Dash

\chapter{Limpeza de dados}\label{cap_trabalho_academico}

% ---
% ----------------------------------------------------------
% Finaliza a parte no bookmark do PDF
% para que se inicie o bookmark na raiz
% e adiciona espaço de parte no Sumário
% ----------------------------------------------------------
\phantompart

% Conclusão
\chapter{Conclusão}


% ELEMENTOS PÓS-TEXTUAIS
\postextual

% Referências bibliográficas
\bibliography{abntex2-modelo-references}

% APENDICES (apendices.tex)
% Pode-se ter múltiplos arquivos, um para cada apêndice
%\begin{apendicesenv}
%\input{apendices}
%\end{apendicesenv}

% ANEXOS (anexos.tex)
% Pode-se ter múltiplos arquivos, um para cada apêndice
%\begin{anexosenv}
%\input{anexos}
%\end{anexosenv}

\end{document}
